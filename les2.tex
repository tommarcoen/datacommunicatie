\documentclass[article,a4paper,12pt]{article}
\usepackage[T1]{fontenc}
\usepackage{libertine}
\usepackage[scaled=.8]{beramono}
\begin{document}

\title{Les 2}
\author{Tom Marcoen}
\date{17 december 2021}
\maketitle


\section{Oefening: VLAN's}

\begin{itemize}
\item Neem twee desktop computers en verbind deze met een Cisco Catalyst 2960-switch.
   \begin{itemize}
   \item Heb je een straight of crossoverkabel nodig?
   \end{itemize}
\item Geef beide computers een IP-adres in hetzelfde netwerk. Kies zelf het netwerkadres maar gebruik het subnet mask 255.255.255.0. Het is niet nodig een default gateway in te vullen.
\item Ping vanaf de eerste computer het IP-adres van de tweede computer.
\item Ping vanaf de tweede computer het IP-adres van de eerste computer.
\item Log nu in op de switch met behulp van PuTTY en maak twee VLAN's aan: een voor de HR-afdeling en een tweede voor het IT operations team.
   \begin{itemize}
   \item Het VLAN-nummer wordt mag je zelf vrij kiezen.
   \item Enkel het nummer is belangrijk. De naam mag je ook vrij kiezen maar kan best een goede omschrijving bevatten.
   \item De VLAN-naam mag spaties bevatten maar dan moet je de naam tussen quotes zetten.
   \item Je kan best eventuele spaties door streeptjes of underscores vervangen en een maximale lengte van 30 tekens hanteren.
   \end{itemize}
\item Plaats de interface van de eerste computer in het HR-VLAN en de interface van de tweede computer in het VLAN voor IT operations.
\item Ping opnieuw van de eerste computer naar de tweede computer en omgekeerd.
\end{itemize}


\section{Communicatie tussen (virtuele) netwerken: de router}

\begin{itemize}
\item Een Cisco router gebruikt ook Cisco IOS dus een router heeft ook de user exec en privileged exec modes.
\item In tegenstelling tot een switch, zijn de interfaces van een router standaard uitgeschakeld.
\item Elke interface van een router moet een IP-adres geconfigureerd hebben om te werken.
\item Elke interface van een router moet zich in een ander netwerk bevinden.
\item Een router is het doorgeefluik voor pakketjes tussen verschillende netwerken.
\end{itemize}

\begin{verbatim}
# configure terminal
(config)# interface Gi0/0/0
(config-if)# no shutdown
(config-if)# ip address 192.168.1.1 255.255.255.0
(config-if)# interface Gi0/0/1
(config-if)# no shutdown
(config-if)# ip address 192.168.2.254 255.255.255.0
(config-if)# end
#show ip interface brief
Interface              IP-Address      OK? Method Status                Protocol 
GigabitEthernet0/0/0   192.168.1.1     YES manual up                    down 
GigabitEthernet0/0/1   192.168.2.254   YES manual up                    down 
GigabitEthernet0/0/2   unassigned      YES unset  administratively down down 
Serial0/1/0            unassigned      YES unset  administratively down down 
Serial0/1/1            unassigned      YES unset  administratively down down 
Vlan1                  unassigned      YES unset  administratively down down
\end{verbatim}




\section{Uitbreiding van de oefening}
\begin{itemize}
\item Geef beide computers IP-adressen in verschillende netwerken.
\item Geef beide computers als default gateway het IP-adres van de router.
\item Ping opnieuw tussen beide computers.
\end{itemize}

\begin{verbatim}
# show ip route connected 
 C   192.168.1.0/24  is directly connected, GigabitEthernet0/0/0
 C   192.168.2.0/24  is directly connected, GigabitEthernet0/0/1
\end{verbatim}



\section{Meerdere VLAN's over \'e\'en kabel}
\begin{itemize}
\item Eén netwerkkabel tussen de router en de switch per VLAN is geen schaalbare oplossing
\item Een trunk interface is een speciale interface waar meerdere (of alle) VLAN's over heen gaan. 
\end{itemize}

\begin{verbatim}
switch# configure terminal
switch(config)# interface GigabitEthernet0/1
switch(config-if)# switchport mode trunk
switch(config-if)# end
\end{verbatim}

De router moet dan ook op een andere manier geconfigureerd worden.

\end{document}

